\documentclass[]{article}
\usepackage{lmodern}
\usepackage{amssymb,amsmath}
\usepackage{ifxetex,ifluatex}
\usepackage{fixltx2e} % provides \textsubscript
\ifnum 0\ifxetex 1\fi\ifluatex 1\fi=0 % if pdftex
  \usepackage[T1]{fontenc}
  \usepackage[utf8]{inputenc}
\else % if luatex or xelatex
  \ifxetex
    \usepackage{mathspec}
  \else
    \usepackage{fontspec}
  \fi
  \defaultfontfeatures{Ligatures=TeX,Scale=MatchLowercase}
\fi
% use upquote if available, for straight quotes in verbatim environments
\IfFileExists{upquote.sty}{\usepackage{upquote}}{}
% use microtype if available
\IfFileExists{microtype.sty}{%
\usepackage{microtype}
\UseMicrotypeSet[protrusion]{basicmath} % disable protrusion for tt fonts
}{}
\usepackage[margin=1in]{geometry}
\usepackage{hyperref}
\hypersetup{unicode=true,
            pdftitle={Problem Set \#1},
            pdfauthor={Daniel Kent for w241 Experiments and Causality},
            pdfborder={0 0 0},
            breaklinks=true}
\urlstyle{same}  % don't use monospace font for urls
\usepackage{longtable,booktabs}
\usepackage{graphicx,grffile}
\makeatletter
\def\maxwidth{\ifdim\Gin@nat@width>\linewidth\linewidth\else\Gin@nat@width\fi}
\def\maxheight{\ifdim\Gin@nat@height>\textheight\textheight\else\Gin@nat@height\fi}
\makeatother
% Scale images if necessary, so that they will not overflow the page
% margins by default, and it is still possible to overwrite the defaults
% using explicit options in \includegraphics[width, height, ...]{}
\setkeys{Gin}{width=\maxwidth,height=\maxheight,keepaspectratio}
\IfFileExists{parskip.sty}{%
\usepackage{parskip}
}{% else
\setlength{\parindent}{0pt}
\setlength{\parskip}{6pt plus 2pt minus 1pt}
}
\setlength{\emergencystretch}{3em}  % prevent overfull lines
\providecommand{\tightlist}{%
  \setlength{\itemsep}{0pt}\setlength{\parskip}{0pt}}
\setcounter{secnumdepth}{0}
% Redefines (sub)paragraphs to behave more like sections
\ifx\paragraph\undefined\else
\let\oldparagraph\paragraph
\renewcommand{\paragraph}[1]{\oldparagraph{#1}\mbox{}}
\fi
\ifx\subparagraph\undefined\else
\let\oldsubparagraph\subparagraph
\renewcommand{\subparagraph}[1]{\oldsubparagraph{#1}\mbox{}}
\fi

%%% Use protect on footnotes to avoid problems with footnotes in titles
\let\rmarkdownfootnote\footnote%
\def\footnote{\protect\rmarkdownfootnote}

%%% Change title format to be more compact
\usepackage{titling}

% Create subtitle command for use in maketitle
\newcommand{\subtitle}[1]{
  \posttitle{
    \begin{center}\large#1\end{center}
    }
}

\setlength{\droptitle}{-2em}

  \title{Problem Set \#1}
    \pretitle{\vspace{\droptitle}\centering\huge}
  \posttitle{\par}
    \author{Daniel Kent for w241 Experiments and Causality}
    \preauthor{\centering\large\emph}
  \postauthor{\par}
      \predate{\centering\large\emph}
  \postdate{\par}
    \date{\begin{verbatim}
oday
\end{verbatim}}


\begin{document}
\maketitle

\section{1. Potential Outcomes
Notation}\label{potential-outcomes-notation}

\begin{itemize}
\item
  Explain the notation \(Y_{i}(1)\).
\item
  Explain the notation \(E[Y_{i}(1)|d_{i}=0]\).
\item
  Explain the difference between the notation \(E[Y_{i}(1)]\) and the
  notation \(E[Y_{i}(1)|d_{i}=1]\). (Extra credit)
\item
  Explain the difference between the notation \(E[Y_{i}(1)|d_{i}=1]\)
  and the notation \(E[Y_{i}(1)|D_{i}=1]\). Use exercise 2.7 from FE to
  give a concrete example of the difference.
\end{itemize}

\section{2. Potential Outcomes
Practice}\label{potential-outcomes-practice}

Use the values in the following table to illustrate that
\(E[Y_{i}(1)] - E[Y_{i}(0)] = E[Y_{i}(1) - Y_{i}(0)]\).

\begin{longtable}[]{@{}llll@{}}
\toprule
\begin{minipage}[b]{0.20\columnwidth}\raggedright\strut
\strut
\end{minipage} & \begin{minipage}[b]{0.16\columnwidth}\raggedright\strut
\(Y_{i}(0)\)\strut
\end{minipage} & \begin{minipage}[b]{0.16\columnwidth}\raggedright\strut
\(Y_{i}(1)\)\strut
\end{minipage} & \begin{minipage}[b]{0.16\columnwidth}\raggedright\strut
\(\tau_i\)\strut
\end{minipage}\tabularnewline
\midrule
\endhead
\begin{minipage}[t]{0.20\columnwidth}\raggedright\strut
Individual 1\strut
\end{minipage} & \begin{minipage}[t]{0.16\columnwidth}\raggedright\strut
5\strut
\end{minipage} & \begin{minipage}[t]{0.16\columnwidth}\raggedright\strut
6\strut
\end{minipage} & \begin{minipage}[t]{0.16\columnwidth}\raggedright\strut
1\strut
\end{minipage}\tabularnewline
\begin{minipage}[t]{0.20\columnwidth}\raggedright\strut
Individual 2\strut
\end{minipage} & \begin{minipage}[t]{0.16\columnwidth}\raggedright\strut
3\strut
\end{minipage} & \begin{minipage}[t]{0.16\columnwidth}\raggedright\strut
8\strut
\end{minipage} & \begin{minipage}[t]{0.16\columnwidth}\raggedright\strut
5\strut
\end{minipage}\tabularnewline
\begin{minipage}[t]{0.20\columnwidth}\raggedright\strut
Individual 3\strut
\end{minipage} & \begin{minipage}[t]{0.16\columnwidth}\raggedright\strut
10\strut
\end{minipage} & \begin{minipage}[t]{0.16\columnwidth}\raggedright\strut
12\strut
\end{minipage} & \begin{minipage}[t]{0.16\columnwidth}\raggedright\strut
2\strut
\end{minipage}\tabularnewline
\begin{minipage}[t]{0.20\columnwidth}\raggedright\strut
Individual 4\strut
\end{minipage} & \begin{minipage}[t]{0.16\columnwidth}\raggedright\strut
5\strut
\end{minipage} & \begin{minipage}[t]{0.16\columnwidth}\raggedright\strut
5\strut
\end{minipage} & \begin{minipage}[t]{0.16\columnwidth}\raggedright\strut
0\strut
\end{minipage}\tabularnewline
\begin{minipage}[t]{0.20\columnwidth}\raggedright\strut
Individual 5\strut
\end{minipage} & \begin{minipage}[t]{0.16\columnwidth}\raggedright\strut
10\strut
\end{minipage} & \begin{minipage}[t]{0.16\columnwidth}\raggedright\strut
8\strut
\end{minipage} & \begin{minipage}[t]{0.16\columnwidth}\raggedright\strut
-2\strut
\end{minipage}\tabularnewline
\bottomrule
\end{longtable}

\section{3. Conditional Expectations}\label{conditional-expectations}

Consider the following table:

\begin{longtable}[]{@{}llll@{}}
\toprule
\begin{minipage}[b]{0.20\columnwidth}\raggedright\strut
\strut
\end{minipage} & \begin{minipage}[b]{0.16\columnwidth}\raggedright\strut
\(Y_{i}(0)\)\strut
\end{minipage} & \begin{minipage}[b]{0.16\columnwidth}\raggedright\strut
\(Y_{i}(1)\)\strut
\end{minipage} & \begin{minipage}[b]{0.16\columnwidth}\raggedright\strut
\(\tau_i\)\strut
\end{minipage}\tabularnewline
\midrule
\endhead
\begin{minipage}[t]{0.20\columnwidth}\raggedright\strut
Individual 1\strut
\end{minipage} & \begin{minipage}[t]{0.16\columnwidth}\raggedright\strut
10\strut
\end{minipage} & \begin{minipage}[t]{0.16\columnwidth}\raggedright\strut
15\strut
\end{minipage} & \begin{minipage}[t]{0.16\columnwidth}\raggedright\strut
5\strut
\end{minipage}\tabularnewline
\begin{minipage}[t]{0.20\columnwidth}\raggedright\strut
Individual 2\strut
\end{minipage} & \begin{minipage}[t]{0.16\columnwidth}\raggedright\strut
15\strut
\end{minipage} & \begin{minipage}[t]{0.16\columnwidth}\raggedright\strut
15\strut
\end{minipage} & \begin{minipage}[t]{0.16\columnwidth}\raggedright\strut
0\strut
\end{minipage}\tabularnewline
\begin{minipage}[t]{0.20\columnwidth}\raggedright\strut
Individual 3\strut
\end{minipage} & \begin{minipage}[t]{0.16\columnwidth}\raggedright\strut
20\strut
\end{minipage} & \begin{minipage}[t]{0.16\columnwidth}\raggedright\strut
30\strut
\end{minipage} & \begin{minipage}[t]{0.16\columnwidth}\raggedright\strut
10\strut
\end{minipage}\tabularnewline
\begin{minipage}[t]{0.20\columnwidth}\raggedright\strut
Individual 4\strut
\end{minipage} & \begin{minipage}[t]{0.16\columnwidth}\raggedright\strut
20\strut
\end{minipage} & \begin{minipage}[t]{0.16\columnwidth}\raggedright\strut
15\strut
\end{minipage} & \begin{minipage}[t]{0.16\columnwidth}\raggedright\strut
-5\strut
\end{minipage}\tabularnewline
\begin{minipage}[t]{0.20\columnwidth}\raggedright\strut
Individual 5\strut
\end{minipage} & \begin{minipage}[t]{0.16\columnwidth}\raggedright\strut
10\strut
\end{minipage} & \begin{minipage}[t]{0.16\columnwidth}\raggedright\strut
20\strut
\end{minipage} & \begin{minipage}[t]{0.16\columnwidth}\raggedright\strut
10\strut
\end{minipage}\tabularnewline
\begin{minipage}[t]{0.20\columnwidth}\raggedright\strut
Individual 6\strut
\end{minipage} & \begin{minipage}[t]{0.16\columnwidth}\raggedright\strut
15\strut
\end{minipage} & \begin{minipage}[t]{0.16\columnwidth}\raggedright\strut
15\strut
\end{minipage} & \begin{minipage}[t]{0.16\columnwidth}\raggedright\strut
0\strut
\end{minipage}\tabularnewline
\begin{minipage}[t]{0.20\columnwidth}\raggedright\strut
Individual 7\strut
\end{minipage} & \begin{minipage}[t]{0.16\columnwidth}\raggedright\strut
15\strut
\end{minipage} & \begin{minipage}[t]{0.16\columnwidth}\raggedright\strut
30\strut
\end{minipage} & \begin{minipage}[t]{0.16\columnwidth}\raggedright\strut
15\strut
\end{minipage}\tabularnewline
\begin{minipage}[t]{0.20\columnwidth}\raggedright\strut
Average\strut
\end{minipage} & \begin{minipage}[t]{0.16\columnwidth}\raggedright\strut
15\strut
\end{minipage} & \begin{minipage}[t]{0.16\columnwidth}\raggedright\strut
20\strut
\end{minipage} & \begin{minipage}[t]{0.16\columnwidth}\raggedright\strut
5\strut
\end{minipage}\tabularnewline
\bottomrule
\end{longtable}

Use the values depicted in the table above to complete the table below.

\newpage

\begin{longtable}[]{@{}lllll@{}}
\toprule
\begin{minipage}[b]{0.19\columnwidth}\raggedright\strut
\(Y_{i}(0)\)\strut
\end{minipage} & \begin{minipage}[b]{0.06\columnwidth}\raggedright\strut
15\strut
\end{minipage} & \begin{minipage}[b]{0.06\columnwidth}\raggedright\strut
20\strut
\end{minipage} & \begin{minipage}[b]{0.06\columnwidth}\raggedright\strut
30\strut
\end{minipage} & \begin{minipage}[b]{0.32\columnwidth}\raggedright\strut
Marginal \(Y_{i}(0)\)\strut
\end{minipage}\tabularnewline
\midrule
\endhead
\begin{minipage}[t]{0.19\columnwidth}\raggedright\strut
10\strut
\end{minipage} & \begin{minipage}[t]{0.06\columnwidth}\raggedright\strut
n: \%:\strut
\end{minipage} & \begin{minipage}[t]{0.06\columnwidth}\raggedright\strut
n: \%:\strut
\end{minipage} & \begin{minipage}[t]{0.06\columnwidth}\raggedright\strut
n: \%:\strut
\end{minipage} & \begin{minipage}[t]{0.32\columnwidth}\raggedright\strut
\strut
\end{minipage}\tabularnewline
\begin{minipage}[t]{0.19\columnwidth}\raggedright\strut
15\strut
\end{minipage} & \begin{minipage}[t]{0.06\columnwidth}\raggedright\strut
n: \%:\strut
\end{minipage} & \begin{minipage}[t]{0.06\columnwidth}\raggedright\strut
n: \%:\strut
\end{minipage} & \begin{minipage}[t]{0.06\columnwidth}\raggedright\strut
n: \%:\strut
\end{minipage} & \begin{minipage}[t]{0.32\columnwidth}\raggedright\strut
\strut
\end{minipage}\tabularnewline
\begin{minipage}[t]{0.19\columnwidth}\raggedright\strut
20\strut
\end{minipage} & \begin{minipage}[t]{0.06\columnwidth}\raggedright\strut
n: \%:\strut
\end{minipage} & \begin{minipage}[t]{0.06\columnwidth}\raggedright\strut
n: \%:\strut
\end{minipage} & \begin{minipage}[t]{0.06\columnwidth}\raggedright\strut
n: \%:\strut
\end{minipage} & \begin{minipage}[t]{0.32\columnwidth}\raggedright\strut
\strut
\end{minipage}\tabularnewline
\begin{minipage}[t]{0.19\columnwidth}\raggedright\strut
Marginal \(Y_{i}(1)\)\strut
\end{minipage} & \begin{minipage}[t]{0.06\columnwidth}\raggedright\strut
\strut
\end{minipage} & \begin{minipage}[t]{0.06\columnwidth}\raggedright\strut
\strut
\end{minipage} & \begin{minipage}[t]{0.06\columnwidth}\raggedright\strut
\strut
\end{minipage} & \begin{minipage}[t]{0.32\columnwidth}\raggedright\strut
1.0\strut
\end{minipage}\tabularnewline
\bottomrule
\end{longtable}

\begin{enumerate}
\def\labelenumi{\alph{enumi}.}
\tightlist
\item
  Fill in the number of observations in each of the nine cells;
\item
  Indicate the percentage of all subjects that fall into each of the
  nine cells.
\item
  At the bottom of the table, indicate the proportion of subjects
  falling into each category of \(Y_{i}(1)\).
\item
  At the right of the table, indicate the proportion of subjects falling
  into each category of \(Y_{i}(0)\).
\item
  Use the table to calculate the conditional expectation that
  \(E[Y_{i}(0)|Y_{i}(1) > 15]\).
\item
  Use the table to calculate the conditional expectation that
  \(E[Y_{i}(1)|Y_{i}(0) > 15]\).
\end{enumerate}

\section{4. More Practice with Potential
Outcomes}\label{more-practice-with-potential-outcomes}

Suppose we are interested in the hypothesis that children playing
outside leads them to have better eyesight.

Consider the following population of ten representative children whose
visual acuity we can measure. (Visual acuity is the decimal version of
the fraction given as output in standard eye exams. Someone with 20/20
vision has acuity 1.0, while someone with 20/40 vision has acuity 0.5.
Numbers greater than 1.0 are possible for people with better than
``normal'' visual acuity.)

\begin{longtable}[]{@{}rrr@{}}
\toprule
child & y0 & y1\tabularnewline
\midrule
\endhead
1 & 1.1 & 1.1\tabularnewline
2 & 0.1 & 0.6\tabularnewline
3 & 0.5 & 0.5\tabularnewline
4 & 0.9 & 0.9\tabularnewline
5 & 1.6 & 0.7\tabularnewline
6 & 2.0 & 2.0\tabularnewline
7 & 1.2 & 1.2\tabularnewline
8 & 0.7 & 0.7\tabularnewline
9 & 1.0 & 1.0\tabularnewline
10 & 1.1 & 1.1\tabularnewline
\bottomrule
\end{longtable}

In the table, state \(Y_{i}(1)\) means ``playing outside an average of
at least 10 hours per week from age 3 to age 6,'' and state \(Y_{i}(0)\)
means ``playing outside an average of less than 10 hours per week from
age 3 to age 6.'' \(Y_{i}\) represents visual acuity measured at age 6.

\begin{enumerate}
\def\labelenumi{\alph{enumi}.}
\item
  Compute the individual treatment effect for each of the ten children.
  Note that this is only possible because we are working with
  hypothetical potential outcomes; we could never have this much
  information with real-world data. (We encourage the use of computing
  tools on all problems, but please describe your work so that we can
  determine whether you are using the correct values.)
\item
  In a single paragraph, tell a story that could explain this
  distribution of treatment effects.
\item
  What might cause some children to have different treatment effects
  than others?
\item
  For this population, what is the true average treatment effect (ATE)
  of playing outside.
\item
  Suppose we are able to do an experiment in which we can control the
  amount of time that these children play outside for three years. We
  happen to randomly assign the odd-numbered children to treatment and
  the even-numbered children to control. What is the estimate of the ATE
  you would reach under this assignment? (Again, please describe your
  work.)
\item
  How different is the estimate from the truth? Intuitively, why is
  there a difference?
\item
  We just considered one way (odd-even) an experiment might split the
  children. How many different ways (every possible way) are there to
  split the children into a treatment versus a control group (assuming
  at least one person is always in the treatment group and at least one
  person is always in the control group)?
\item
  Suppose that we decide it is too hard to control the behavior of the
  children, so we do an observational study instead. Children 1-5 choose
  to play an average of more than 10 hours per week from age 3 to age 6,
  while Children 6-10 play less than 10 hours per week. Compute the
  difference in means from the resulting observational data.
\item
  Compare your answer in (h) to the true ATE. Intuitively, what causes
  the difference?
\end{enumerate}

\section{5. Randomization and
Experiments}\label{randomization-and-experiments}

Suppose that a reasearcher wants to investigate whether after-school
math programs improve grades. The researcher randomly samples a group of
students from an elementary school and then compare the grades between
the group of students who are enrolled in an after-school math program
to those who do not attend any such program. Is this an experiment or an
observational study? Why?

\section{6. Lotteries}\label{lotteries}

A researcher wants to know how winning large sums of money in a national
lottery affect people's views about the estate tax. The research
interviews a random sample of adults and compares the attitudes of those
who report winning more than \$10,000 in the lottery to those who claim
to have won little or nothing. The researcher reasons that the lottery
choose winners at random, and therefore the amount that people report
having won is random.

\begin{enumerate}
\def\labelenumi{\alph{enumi}.}
\tightlist
\item
  Critically emvaluate this assumption.
\item
  Suppose the researcher were to restrict the sample to people who had
  played the lottery at least once during the past year. Is it safe to
  assume that the potential outcomes of those who report winning more
  than \$10,000 are identical, in expectation, to those who report
  winning little or nothing?
\end{enumerate}

\emph{Clarifications}

\begin{enumerate}
\def\labelenumi{\arabic{enumi}.}
\tightlist
\item
  Please think of the outcome variable as an individual's answer to the
  survey question ``Are you in favor of raising the estate tax rate in
  the United States?''
\item
  The hint about potential outcomes could be rewritten as follows: Do
  you think those who won the lottery would have had the same views
  about the estate tax if they had actually not won it as those who
  actually did not win it? (That is, is
  \(E[Y_{i}0|D=1] = E[Y_{i}0|D=0]\), comparing what would have happened
  to the actual winners, the \(|D=1\) part, if they had not won, the
  \(Y_{i}(0)\) part, and what actually happened to those who did not
  win, the \(Y_{i}(0)|D=0\) part.) In general, it is just another way of
  asking, ``are those who win the lottery and those who have not won the
  lottery comparable?''
\item
  Assume lottery winnings are always observed accurately and there are
  no concerns about under- or over-reporting.
\end{enumerate}

\section{7. Inmates and Reading}\label{inmates-and-reading}

A researcher studying 1,000 prison inmates noticed that prisoners who
spend at least 3 hours per day reading are less likely to have violent
encounters with prison staff. The researcher recommends that all
prisoners be required to spend at least three hours reading each day.
Let \(d_{i}\) be 0 when prisoners read less than three hours each day
and 1 when they read more than three hours each day. Let \(Y_{i}(0)\) be
each prisoner's PO of violent encounters with prison staff when reading
less than three hours per day, and let \(Y_{i}(1)\) be their PO of
violent encounters when reading more than three hours per day.

In this study, nature has assigned a particular realization of \(d_{i}\)
to each subject. When assessing this study, why might one be hesitant to
assume that \({E[Y_{i}(0)|D_{i}=0] = E[Y_{i}(0)|D_{i}=1]}\) and
\(E{[Y_{i}(1)|D_{i}=0] = E[Y_{i}(1)|D_{i}=1]}\)? In your answer, give
some intuitive explanation in English for what the mathematical
expressions mean.


\end{document}
